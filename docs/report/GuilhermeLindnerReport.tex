\documentclass[12pt, a4paper, openany]{memoir}
\usepackage[top=2.5cm, bottom=2.5cm, left=2.5cm, right=2.5cm]{geometry}
\usepackage{listings}
\usepackage{color}
\usepackage{algorithm2e}
\usepackage[noend]{algpseudocode}
%\usepackage{algpseudocode}
\usepackage{graphicx}
\usepackage[none]{hyphenat}
\usepackage{csvsimple}
%\usepackage{placeins}
\usepackage{chngcntr}
\usepackage{amsmath} %matrix
\usepackage{indentfirst} %indent the first paragraph after new section
%\usepackage[style=authoryear]{biblatex}
\usepackage{tikz} % flowchart
\usetikzlibrary{shapes,arrows} % flowchart
\usepackage{hyperref} % links
\usepackage{subcaption}
\usepackage[section,above,below]{placeins} % floatbarrier em tudo?
%\usepackage{fancyhdr}
%\pagestyle{fancy} 
%\fancyhf{}

\setsecnumdepth{subsection} %show number in subsection
\maxtocdepth{subsection} %show subsection in toc

\tikzstyle{decision} = [diamond, draw, fill=blue!20, 
text width=4.5em, text badly centered, node distance=3cm, inner sep=0pt]
\tikzstyle{block} = [rectangle, draw, fill=blue!20, 
text width=5em, text centered, rounded corners, minimum height=4em]
\tikzstyle{line} = [draw, -latex']
\tikzstyle{cloud} = [draw, ellipse,fill=red!20, node distance=3cm,
minimum height=2em]


\newcommand{\norm}[1]{\left\lVert#1\right\rVert}


\title{ \normalsize \textsc{Master 1 - Final year project}
	\\ [2.0cm]
	\hrule
	\vspace{0.5cm}
	\LARGE \textbf{\uppercase{Vortex detection and fitting in turbulent flows}} \\ [0.5cm]
	\hrule
	\vspace{0.5cm}
	\normalsize  \vspace*{5\baselineskip}}


\date{\vfill May - 2017}

\author{
	\Large Guilherme Lindner \\ [1.0cm]
	Research Advisors: \\
	Jean Marc Foucaut \\
	Jean-Philippe Laval  \\ [1.0cm]
	International Master on Turbulence\\ [0.2cm]
	Ecole Centrale, Lille \\}

\begin{document}
	
	\maketitle
	\thispagestyle{empty}
	\let\cleardoublepage\clearpage
	\frontmatter
	\begin{abstract}
		In this project, a tool is developed in Python to detect and evaluate the coherent structures of a velocity field. The workflow is based on the thesis of Sophie Herpin \cite{herpin2009} pubished in 2009.
		This tool comprehends the reading of NetCDF files and applying a method for detecting the vortices, like swirling strength of Q criterion. After this, all the vortices found are fitted to the Lamb-Oseen vortex using a non-linear least squares method and the correlation between the model and original data is evaluated. If the correlation is higher than 0.75, the vortex is accepted, otherwise is reject. The analyzed data can come from Direct Numerical Simulation (DNS) or Particle Image Velocimetry (PIV).
	\end{abstract}
	\newpage
	\tableofcontents
	\newpage
%	\thispagestyle{empty}
	\listoftables
	\newpage
	\listoffigures

    \mainmatter

\chapter*{Introduction}
\addcontentsline{toc}{chapter}{Introduction} %\markboth{INTRODUCTION}{}
Coherent structures and vortices sometimes are misunderstood for not having a clear definition. In general ways, any form of pattern arising in the flow that has an effect on it is considered a coherent structure. Meanwhile, the vortices are commonly explained as the existence of some kind of rotation in the flow. The difficulty on identifying these vortices it's in the fact that is not so clear to measure the extension of the vortex from its center of rotation. All of this becomes more complicated when we have the interaction between multiples coherent structures. According to Shawn \textit{et al} (2005) \cite{shawn2005} the definition is:
\begin{quote}
	\textit{"It is generally accepted that flows with general time dependence admit emergent patterns which influence the transport of tracers, those structures are often generically refered to as Coherent Structures"}
\end{quote}

In short we can say that vortices are coherent structures, but the inverse is most of the times true, but not a rule.



\chapter{Methodology}

The methodology used in this work to characterize the coherent structures are divided in three main steps: the calculation of a local function to detect the vortex; the localization of the maxima of this function; and the adjustment (fitting) of this field to the proposed model.

\section{Detection methods}
In this section, the detection methods implemented in the code for vortex identification are presented. These methods are based on the velocity gradient tensor, $\overline{D}$, that can be written as:

\begin{equation}
D_{ij} = \frac{\partial u_i}{\partial x_j}
\end{equation}

As this is a second order tensor, it can be decomposed into a symmetric and anti-symmetric part, $D_{ij} = S_{ij} + \Omega_{ij}$ where:

\begin{equation}
S_{ij} = \frac{1}{2} \left(\frac{\partial u_i}{\partial x_j} + \frac{\partial u_j}{\partial x_i}\right)
\end{equation}

\begin{equation}
\Omega_{ij} = \frac{1}{2} \left(\frac{\partial u_i}{\partial x_j} - \frac{\partial u_j}{\partial x_i}\right)
\end{equation}
$S_{ij}$ is known as the rate-of-strain tensor and $\Omega_{ij}$ is the vorticity tensor.

The characteristic equation for $\nabla u$ is given by

\begin{equation}
\lambda^3 + P \lambda^2 + Q \lambda + R = 0
\end{equation}
where P, Q and R are the three invariants of the velocity gradient tensor. Using the decomposition into symmetric and anti-symmetric parts, these invariants can be expressed as:
\begin{equation}
P = -tr(\bar{D})
\end{equation}

\begin{equation}
Q = \frac{1}{2} (tr(\bar{D})^2 -tr(\bar{D}^2)) = \frac{1}{2} (\norm{\Omega}^2 - \norm{S}^2)
\end{equation}

\begin{equation}
R = -det(\bar{D})
\end{equation}

\subsection{Q criterion}

The Q criterion proposed by Hung \textit{et al} (1988) identifies the vortices as flow regions with positive second invariant of $\nabla u$. An additional condition is that the pressure in the eddy region should to be lower than the ambient pressure. Chakraborty \textit{et al}(2005) \cite{chakra2005} quoted "in an incompressible flow Q is a local measure of the excess rotation rate relative to the strain rate".

In practical terms, the vortex is detected in case of the second invariant $Q > 0$.

\subsection{$\Delta$ criterion}

Chong \textit{et al} (1990) \cite{chong1990} define a vortex core to be the region where $\nabla v$ has complex eigenvalues. In order to determine if the eigenvalues are complex, we examine the discriminant of the characteristic equation, considering the flow incompressible (P = 0).

\begin{equation}
\Delta = \left(\frac{Q}{3}\right)^3 + \left(\frac{R}{2}\right)^2 > 0
\end{equation}


\subsection{Swirling strength criterion}

The swirling strength criterion ($\lambda_{ci}$) was developed by Zhou \textit{et al} (1999) \cite{zhou1999}. It defines a vortex core to be the region where $\nabla v$ has complex eigenvalues. It is based on the idea that the velocity gradient tensor in Cartesian coordinates can be decomposed as:

\begin{equation}
\nabla u = [\bar{\nu_r} \bar{\nu_{cr}} \bar{\nu_{ci}}]^T
\left[\begin{array}{ccc}
\lambda_r & 0 & 0 \\
0 & \lambda_{cr} & \lambda{ci} \\
0 & -\lambda{ci} & \lambda{cr} \end{array}\right]
[\bar{\nu_r} \bar{\nu_{cr}} \bar{\nu_{ci}}]^T
\end{equation} 
where $\lambda_r$ is the real eigenvalue, related to the eigenvector $\bar{\nu_r}$, and the complex conjugate pair of complex eigenvalues is $\lambda_{cr}  \pm i\lambda_{ci}$, related to the eigenvectors $\bar{\nu_r} \pm i\bar{\nu_{ci}}$. The strength of this swirling motion can be quantified by $\lambda_{ci}$ , called the local swirling strength of the vortex. The threshold for $\lambda_{ci}$ is not well-defined, but theoretically any value greater than zero should be considered a vortex. Experimental results \cite{zhou1999} shows that $\lambda_{ci} \geq \epsilon > 0$ give smoother results.



\section{Localization of the extrema}

To have smooth results on the swirling strength, we apply a normalization of the field. The swirling strength is divided by the wall-normal profile of its RMS value:

\begin{equation}
\bar{\lambda}_{ci}(x_{1/3},x_2) = \frac{\lambda_{ci}(x_{1/3},x_2)}{\lambda_{ci,RMS}(x_2)}
\end{equation}

Then the local maxima of the detection can be identified. The normalization is not required for the HIT cases, it is only used when we have an non-homogeneous direction.

\begin{figure}[h]
	\centering
	\includegraphics[trim=0 130 0 130 ,clip, width=\textwidth]{figure/PIVnonnormalized.pdf}
	\caption{Original swirling strength field}
	\label{fig:nonnorm}
\end{figure}

\begin{figure}[h]
	\centering
	\includegraphics[trim=0 130 0 130 , clip, width=\textwidth]{figure/PIVnormalized.pdf}
	\caption{Normalized swirling strength field}
	\label{fig:norm}
\end{figure}

In figure \ref{fig:nonnorm} we see the original swirling strength field, where 104 vortices were found mostly near the wall. In figure \ref{fig:norm} we show the normalized swirling strength field, now with 202 vortices found. 

\section{Fitting of coherent structures}

The correlation coefficient between the fitted model and the velocity field is calculated according to equation.

\begin{equation}
R(model/data) = \left( \frac{\langle (\vec{u}_{data}-\vec{u}_c).(\vec{u}_{model}-\vec{u}_c)\rangle }
{\sqrt{\langle (\vec{u}_{data}-\vec{u}_c)^2\rangle} \sqrt{\langle (\vec{u}_{model}-\vec{u}_c)^2\rangle}} \right)^{1/2}
\end{equation}

\subsection{Lamb-Oseen vortex}

The Lamb-Oseen vortex is a mathematical model for the flow velocity in the circumferential direction ($\theta$), shown in equation \ref{eq:oseenDecay}. It models a line vortex that decays due to viscosity.

\begin{equation}
\label{eq:oseenDecay}
\vec{u}_\theta(r,t) = \frac{\Gamma}{2\pi r} \left( 1 - exp \left( -\left(\frac{r}{r_0(t)}\right)^2\right)\right) \vec{e}_{\theta}
\end{equation}
where $r$ is the radius, $r_0 = \sqrt{4 \nu t}$ is the core radius of vortex, $\nu$ is the viscosity and $\Gamma$ is the circulation contained in the vortex. 

In this work we are dealing with a time-independent flow, so we have no decaying due to viscosity. And since the coherent structures are in movement, we add the convective velocity to the Lamb-Oseen vortex model shown in equation \ref{eq:oseen}.  

\begin{equation}
\label{eq:oseen}
\vec{u}(r,\theta) = \vec{u}_c + \frac{\Gamma}{2\pi r} \left( 1 - exp \left( -\left(\frac{r}{r_0}\right)^2\right)\right) \vec{e}_{\theta}
\end{equation}

\subsection{Non-linear least squares}

\subsubsection{Levenberg Marquardt method}

The Levenberg–Marquardt algorithm, also known as the damped least-squares method, is used to solve non-linear least squares problems. These minimization problems arise especially in least squares curve fitting.

\begin{equation}
\chi^2 = \sum_{i=1}^N \left[ \frac{y_i - \sum_{k=1}^M a_k X_k (x_i)}{\sigma i} \right]^2
\end{equation}

\begin{equation}
\alpha_{kl} = \sum_{i=1}^N \frac{1}{\sigma_i^2} \left[ \frac{\partial y(x_i;a)}{\partial a_k} \frac{\partial y(x_i;a)}{\partial a_l} \right]
\end{equation}

\subsubsection{Powell's dogleg method}

The Powell's method is an algorithm for finding a local minimum of a function. This function doesn't need to be differentiable and no derivatives are taken. It does this using a combination of Newton's method and the steepest descent method. This is a so-called trust region method. This means that every step moves the current point to within a finite region. This makes the method more stable than Newton's method.

\section{Programming}

The language chosen for developing the detection tools is Python 3.6 (Python Software Foundation, https://www.python.org/). This choice was made for the following reasons:
\begin{itemize}
	\item Readability \\
	Python's syntax is easy to read, very close to pseudocode itself.
	\item Access to low level programming languages \\
	We can easily add C or Fortran code o be run inside Python's code, reusing some useful libraries and tools.
	\item Language interoperability \\
	MATLAB functions can be called from Python through the MATLAB engine and even other languages.
	\item Documentation system \\
	One helpful feature for scientific programming is the ability to put LaTeX equations and plots directly in code documentation, by using the docstrings.
	\item Available libraries \\
	Python has an impressive standard library packaged with Python. For scientific uses, the most well known are NumPy and SciPy.
\end{itemize} 

\newpage
\subsection{Code Structure}

The following files are present on the code, allowing the easy implementation of new methods, tools or scheemes.

\begin{itemize}
	\item classes.py : definition of the data type, i.e. DNS, PIV, ...
	\item detection.py : detect the vortices by one of the methos (Q criterion, $\lambda_2$ criterion or swirling strenth)
	\item fitting.py : performs the fit to oseen vortex
	\item plot.py : plot the desired results, like scalar fiels, vectors, ...
	\item schemes.py : differencing schemes (2nd and 4th order)
	\item testoseen.py : tests for fitting performance
	\item tools.py : normalization, vortex window definition, local maxima detection, ... 
	\item vortexfitting.py : main program
\end{itemize}

The code uses the version control system Git, and it's hosted and fully available at \url{https://github.com/guilindner/VortexFitting}. 

\newpage
\subsection{Routine}

Here we describe in the figure \ref{fig:mainprogramfc} the routine done by the main program, as well the details of the fitting routine on figure \ref{fig:fittingfc}. Currently the user must specify on the \textit{classes.py} file if the data have one homogeneous direction or not, so the proper normalization and smoothing can be done by the routine if required.

\begin{figure}[h!]
	\centering
	\includegraphics[trim=150 150 150 50 ,clip, scale=0.8]{figure/vortexFittingFlow.pdf}
	\caption{flowchart of the main program.}
	\label{fig:mainprogramfc}
\end{figure}

\newpage
\begin{figure}[h!]
	\centering
	\includegraphics[trim=50 320 50 0 ,clip, scale=0.8]{figure/FittingFlow.pdf}
	\caption{Detailed flowchart of the "Fitting of the model".}
	\label{fig:fittingfc}
\end{figure}

This iteration loop allows the fitting window to be adjusted according to the core radius of the vortex, and simultaneously change the center position of the vortex if needed. 

\chapter{Validation}
In this chapter we will use some controlled cases to test the overall performance of the code.

\section{Detection}
Here we will test the three detection methods, Q criterion, $\Delta$ criterion and swirling strength.

For the derivative scheme, we will use the 4th order accurate scheme and 2nd order (details on Annnex \ref{annex:finite}). According to Raffel \textit{et al} (1998), the 2nd order scheme is recommended because it minimizes the propagation of measurement noise.   

The tested case is an Homogeneous Isotropic Turbulence (HIT) of a cube, obtained from Direct Numerical Simulation (DNS) calculations.

\begin{table}[h]
	\centering
	\caption{Vortices found, detection X schemes}
	\vspace{10px}
	\label{tb:detection}
	\begin{tabular}{l|l|l}
		         & 2nd order      & 4th order  \\
		\hline
		Q criterion  & 403  & 398  \\
		$\Delta$ criterion & 462  & 462 \\
		Swirling strength    & 413  & 409
	\end{tabular}
\end{table}

\section{Fitting}

In this section we will test the Oseen Vortex model fitting under different scenarios. For the first comparison a standard Oseen Vortex is created to be an exact match and for the second set of tests we add a noise to the original vortex field.

Figures \ref{fig:fittingtests} and \ref{fig:fittingtestsnoise} shows 4 different cases:
\begin{enumerate}[(a)]
	\item core radius = 0.2, $\Gamma$ = 10, displacement x and y = 0.0
	\item core radius = 0.2, $\Gamma$ = 10, displacement x and y = 0.2
	\item core radius = 0.9, $\Gamma$ = 40, displacement x and y = 0.0
	\item core radius = 0.9, $\Gamma$ = 40, displacement x and y = 0.2  
\end{enumerate}

\begin{figure}[h!]
	\centering
	\begin{subfigure}[b]{0.45\textwidth}
		\centering
		\includegraphics[trim=40 20 40 20 ,clip, width=\textwidth]{figure/test_02_10.pdf}
		\caption{r = 0.2, Gamma = 10}
	\end{subfigure}
	\begin{subfigure}[b]{0.45\textwidth}
		\centering
		\includegraphics[trim=40 20 40 20 ,clip, width=\textwidth]{figure/test_02_10_02.pdf}
		\caption{r = 0.2, Gamma = 10, shift = 0.2}
	\end{subfigure}
	\begin{subfigure}[b]{0.45\textwidth}
		\centering
		\includegraphics[trim=40 20 40 20 ,clip, width=\textwidth]{figure/test_09_40.pdf}
		\caption{r = 0.9, Gamma = 40}
	\end{subfigure}
	\begin{subfigure}[b]{0.45\textwidth}
		\centering
		\includegraphics[trim=40 20 40 20 ,clip, width=\textwidth]{figure/test_09_40_02.pdf}
		\caption{r = 0.9, Gamma = 40, shift = 0.2}
	\end{subfigure}
	\caption{Fitting test}
	\label{fig:fittingtests}
\end{figure}

\begin{table}[h]
	\centering
	\caption{Fitting test results}
	\vspace{10px}
	\label{tb:fittingtest}
	\begin{tabular}{l|l|l|l|l}
		            & a      & b & c & d \\
		\hline
		radius      & 0.2000   & 0.2000 & 0.9000 & 0.8999  \\
		gamma       & 10.0000 & 10.0000 & 40.0000 & 39.9999 \\
		x shift     & 0.0000  & 0.2000 & 0.0000 & 0.2000 \\
		y shift     & 0.0000  & 0.2000 & 0.0000 & 0.2000 \\ 
		correlation & 1.0000   & 1.0000 & 1.0000 & 1.0000 \\
	\end{tabular}
\end{table}

We can see in table \ref{tb:fittingtest} that the guess is exact for all the cases, as the vectors totally overlaps themselves (figure \ref{fig:fittingtests}).

Now we add a perturbation to the initial flow field. Considering a random noise of 30\% on the original field we have the following performance of the fitting presented on table \ref{tb:fittingtestnoise} in respect to the figures \ref{fig:fittingtestsnoise}.

\begin{figure}[h!]
	\centering
	\begin{subfigure}[b]{0.45\textwidth}
		\centering
		\includegraphics[trim=40 20 40 20 ,clip, width=\textwidth]{figure/test_02_10N.pdf}
		\caption{r = 0.2, Gamma = 10}
	\end{subfigure}
	\begin{subfigure}[b]{0.45\textwidth}
		\centering
		\includegraphics[trim=40 20 40 20 ,clip, width=\textwidth]{figure/test_02_10_02N.pdf}
		\caption{r = 0.2, Gamma = 10, shift = 0.2}
	\end{subfigure}
	\begin{subfigure}[b]{0.45\textwidth}
		\centering
		\includegraphics[trim=40 20 40 20 ,clip, width=\textwidth]{figure/test_09_40N.pdf}
		\caption{r = 0.9, Gamma = 40}
	\end{subfigure}
	\begin{subfigure}[b]{0.45\textwidth}
		\centering
		\includegraphics[trim=40 20 40 20 ,clip, width=\textwidth]{figure/test_09_40_02N.pdf}
		\caption{r = 0.9, Gamma = 40, shift = 0.2}
	\end{subfigure}
	\caption{Fitting test with random noise}
	\label{fig:fittingtestsnoise}
\end{figure}

\begin{table}[h]
	\centering
	\caption{Fitting test results with noise}
	\vspace{10px}
	\label{tb:fittingtestnoise}
	\begin{tabular}{l|l|l|l|l}
		            & a      & b & c & d \\
		\hline
		radius      & 0.1917   & 0.2427 & 0.9067 & 0.9169  \\
		gamma       & 10.1863 & 9.9743 & 40.4885 & 40.5987 \\
		x shift     & 0.0035  & 0.1946 & -0.0098 & 0.1953 \\
		y shift     & 0.0028  & 0.1918 & 0.0085 & 0.1995 \\ 
		correlation & 0.9406   & 0.9153 & 0.9895 & 0.9869 \\
	\end{tabular}
\end{table}


\chapter{Results}
Here we show the results of the full fitting procedure of DNS data.

\section{DNS data}
This case is an homogeneous isotropic turbulence (HIT) study, generated by a DNS simulation. Since we have no walls and boundaries, no normalization of the velocity field or smoothing on the swirling field is needed. The scheme used is the second order central scheme and using the swirling strength criteria for detection.

On figure \ref{fig:velocity_field} we show the velocity field of the HIT DNS data. The 3 components are shown, for the a slice in the middle of the box.

\begin{figure}[h!]
	\centering
	\begin{subfigure}[b]{0.45\textwidth}
		\centering
		\includegraphics[trim=80 20 80 20 ,clip,width=\textwidth]{figure/dns_u.pdf}
		\caption{u velocity.}
	\end{subfigure}
	\begin{subfigure}[b]{0.45\textwidth}
		\centering
		\includegraphics[trim=80 20 80 20 ,clip,width=\textwidth]{figure/dns_v.pdf}
		\caption{v velocity}
	\end{subfigure}
	\begin{subfigure}[b]{0.45\textwidth}
		\centering
		\includegraphics[trim=80 20 80 20 ,clip,width=\textwidth]{figure/dns_w.pdf}
		\caption{w velocity}
	\end{subfigure}
	\begin{subfigure}[b]{0.45\textwidth}
		\centering
		\includegraphics[trim=80 20 80 20 ,clip,width=\textwidth]{figure/dns_total.pdf}
		\caption{total velocity}
	\end{subfigure}
	\caption{Velocity field of a 3D HIT DNS}
	\label{fig:velocity_field}
\end{figure}

We notice on figure \ref{fig:quiverDNS1} that the peak of swirling strength does not always match the center of the vortex. That's why we need to iterate over the center of vortex window to get the best correlation.

\begin{figure}[h]
	\centering
	\includegraphics[scale=0.6]{figure/dns_quiver1.pdf}
	\caption{Velocity vectors on top of the swirling strength field}
	\label{fig:quiverDNS1}
\end{figure}

\begin{figure}[h]
	\centering
	\includegraphics[scale=0.6]{figure/dns_detect.pdf}
	\caption{Detected vortices from the DNS data}
	\label{fig:detectionDNS}
\end{figure}

In figure \ref{fig:detectionDNS}, 278 peaks of swirling strength (vortices candidates) were found, where the yellow circles corresponds to the detected vortices rotating clockwise and the green circles for the counter-clockwise rotation. From these vortices, only 23 were accepted as having a correlation higher than 0.75 with the lamb-oseen vortex.

\begin{figure}[h!]
	\centering
	\begin{subfigure}[b]{0.45\textwidth}
		\centering
		\includegraphics[trim=40 20 40 20 ,clip, width=\textwidth]{figure/dns_fit1.pdf}
		\caption{}
	\end{subfigure}
	\begin{subfigure}[b]{0.45\textwidth}
		\centering
		\includegraphics[trim=40 20 40 20 ,clip, width=\textwidth]{figure/dns_fit2.pdf}
		\caption{}
	\end{subfigure}
	\begin{subfigure}[b]{0.45\textwidth}
		\centering
		\includegraphics[trim=40 20 40 20 ,clip, width=\textwidth]{figure/dns_fit3.pdf}
		\caption{}
	\end{subfigure}
	\begin{subfigure}[b]{0.45\textwidth}
		\centering
		\includegraphics[trim=40 20 40 20 ,clip, width=\textwidth]{figure/dns_fit6.pdf}
		\caption{}
	\end{subfigure}
	\caption{Fitted vortices}
	\label{fig:vorticesDNS}
\end{figure}

Figure \ref{fig:vorticesDNS} shows 4 random fitted vortices from the DNS, and table \ref{tb:DNSvortices} describes its characteristics.

\begin{table}[h!]
	\centering
	\caption{DNS vortices characteristics}
	\vspace{10px}
	\label{tb:DNSvortices}
	\begin{tabular}{l|l|l|l|l|l}
		Vortex         & Center(x,y) & core radius    & $\Gamma$   & correlation & mesh window \\
		\hline
		a    & 37, 50   & 0.1226  & 6.0498   & 0.7818 & 6 \\
		b    & 93, 79   & 0.0745  & 7.9175   & 0.8590 & 4 \\
		c    & 130, 36  & 0.0844  & -11.3036 & 0.9305 & 4\\
		d    & 227, 124 & 0.11727 & -13.6144 & 0.9309 & 6
	\end{tabular}
\end{table}
  

%\section{PIV data}

%\begin{figure}[h]
%	\centering
%	\label{fig:detectionPIV}
%	\includegraphics[scale=0.5]{figure/dummy.png}
%	\caption{Detected vortices from the PIV data}
%\end{figure}

\newpage
\bibliographystyle{plain}
\bibliography{biblio}

\appendix

\chapter{Finite difference methods}

\section{Second order accurate}
\label{annex:finite}
Two schemes were selected to be implemented, the 2nd order scheme and the 4th order scheme. The stencils are presented below.

Second order central differencing scheme, on interior cells:
\begin{equation}
F_x(i) = \frac{F(i+1)-F(i-1)}{2 \Delta x}
\end{equation}

Forward at left edge cells:
\begin{equation}
F_x(i) = \frac{F(i+1)-F(i)}{\Delta x}
\end{equation}

Backward at right edge cells:
\begin{equation}
F_x(n) = \frac{F(n)-F(n-1)}{\Delta x}
\end{equation}

\section{Fourth order accurate}

\begin{equation}
F_x(i) = \frac{-F(i+2)+8F(i+1)-8F(i-1)+F(i-2)}{12 \Delta x}
\end{equation}

left edge cells:
\begin{equation}
F_x(i) = \frac{F(i+3)-6F(i+2)+18F(i+1)-10F(i) -3F(i-1)}{12 \Delta x}
\end{equation}

right edge cells:
\begin{equation}
F_x(i) = \frac{3F(i+1)+10F(i)-18F(i-1)+6F(i-2) -F(i-3)}{12 \Delta x}
\end{equation}


\end{document}














